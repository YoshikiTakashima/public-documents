\documentclass{resume} % Use the custom resume.cls style
% Document margins
\usepackage[utf8]{inputenc}
\usepackage[left=0.75in,top=0.6in,right=0.75in,bottom=0.6in]{geometry}
\usepackage[T1]{fontenc}
\usepackage{titling}
\usepackage{atbegshi}
\usepackage{hyperref}
\usepackage{nopageno}
\thispagestyle{empty}
% http://ctan.org/pkg/atbegshi
\AtBeginDocument{\AtBeginShipoutNext{\AtBeginShipoutDiscard}}

% \sloppy

\title{\bf Yoshiki Takashima}
\author{\href{ytakashi@andrew.cmu.edu}{ytakashi@andrew.cmu.edu}}
\date{4720 Forbes Ave, RMCIC-2119C. Pittsburgh PA 15213}

\begin{document}
\maketitle

\begin{rSection}{Research Interest}
  I am interested in helping developers build more robust software by
  leveraging automated, language-aware solutions for testing and
  verification: recent work focuses on testing and verifying Rust
  libraries.
\end{rSection}

\begin{rSection}{Education}
  \begin{rSubsection}{Carnegie Mellon University}{$2019 -$
      Present}{PhD Student in Electrical and Computer
      Engineering}{Pittsburgh, PA}
  \item Co-Advised by: Prof. Limin Jia and Prof. Corina P\u{a}s\u{a}reanu.
  % \item QPA 3.62/4.00
  \end{rSubsection}

  \begin{rSubsection}{UC San Diego}{$2017 -
      2019$}{
      \href{https://catalog.ucsd.edu/curric/MATH-ug.html}
      {BS Mathematics - Comp. Sci. (MA30)}}{La Jolla, CA}
  \item GPA: $3.95/4.00$
  \item Credits transferred from Santa Monica College and West Los
    Angeles College.
  \end{rSubsection}

  % \begin{rSubsection}{Santa Monica College}{$2015 -
  %     2017$}{IGETC Transfer Certificate}{Santa Monica, CA}
  % \item GPA: $3.86/4.00$
  % \item High school concurrent enrollment.
  % \end{rSubsection}

  % \begin{rSubsection}{West Los Angeles College}{$2014$}{Credits
  %     Transferred}{Los Angeles, CA}
  % \item GPA: $4.00/4.00$
  % \item Concurrent enrollment.
  % \end{rSubsection}
\end{rSection}

\begin{rSection}{Publications}

  \begin{rSubsection}{}{}{}{}
  \item \textbf{PropProof: Free Model-Checking Harnesses from PBT}.
    Yoshiki Takashima. ACM Joint European Software Engineering
    Conference and Symposium on the Foundations of Software
    Engineering 2023 \textit{(ESEC/FSE’23 Industry Track)}.

    Code: \href{https://github.com/YoshikiTakashima/propproof}
    {https://github.com/YoshikiTakashima/propproof}

  \item \textbf{Mariposa: Measuring SMT Instability in Automated
      Program Verification}. Yi Zhou, Jay Bosamiya, Yoshiki Takashima,
    Jessica Li, Marijn Heule, Bryan Parno. Formal Methods in
    Computer-Aided Design 2023 \textit{(FMCAD'23)}

    Code: \href{https://github.com/secure-foundations/mariposa}
    {https://github.com/secure-foundations/mariposa}

  \item \textbf{SyRust: Automatic Testing of Rust Libraries
    with Semantic-Aware Program Synthesis}.
    Yoshiki Takashima, Ruben Martins, Limin Jia, and Corina
    S. P\u{a}s\u{a}reanu.  In Proceedings of the 42nd
    ACM SIGPLAN International Conference on Programming Language
    Design and Implementation \textit{(PLDI’21)}.

    Paper, video, and artifact available here:
    \href{https://doi.org/10.1145/3453483.3454084}
    {https://doi.org/10.1145/3453483.3454084}. Discovered bugs lead to
    \href{https://nvd.nist.gov/vuln/detail/CVE-2020-15254}{CVE-2020-15254}

  \item \textbf{VeriSketch: Synthesizing Secure Hardware Designs with
      Timing-Sensitive Information Flow Properties}.  Armaiti
    Ardeshiricham, Yoshiki Takashima (\textbf{presenter}), Sicun Gao,
    Ryan Kastner. In Proceedings of the 2019 ACM SIGSAC Conference on
    Computer and Communications Security~\textit{(CCS'19)}.

    Paper and video available here:
    \href{https://dl.acm.org/doi/abs/10.1145/3319535.3354246}
    {https://doi.org/10.1145/3319535.3354246}
  \end{rSubsection}
\end{rSection}

\begin{rSection}{Employment}

  \begin{rSubsection}{Applied Scientist Intern (Automated Reasoning),
      Amazon Web Services}{Summer $2023$} {}{}
  \item Developed an universal Rust translator with formal guarantees
    of equivalence between the original program and the LLM-translated
    Rust program.
  \end{rSubsection}

  \begin{rSubsection}{Applied Scientist Intern (Automated Reasoning),
      Amazon Web Services}{Summer $2022$} {}{}
  \item Developed PropProof, a library that automatically converts
    Property-Based Testing harnesses into model-checking by replacing
    random values with symbolic ones. \textit{(ESEC/FSE'23 Industry Track)}
  \end{rSubsection}
\end{rSection}


\begin{rSection}{Awards and Fellowships}

  \begin{rSubsection}{Amazon Research Award 2022 CFP: Enabling One-Line Rust
      Verification with Program Synthesis}{}{}{}
  \item Written with advisors Limin Jia and Corina P\u{a}s\u{a}reanu.
  \end{rSubsection}

  \begin{rSubsection}{Prabhu and Poonam Goel Graduate Fellowship}{$2021 - 2022$}{}{}
    \item Awarded by Electrical and Computer Engineering Department.
  \end{rSubsection}
\end{rSection}


\begin{rSection}{Teaching and Service}
  \begin{rSubsection}{Teaching Assistant for 18-636 (Web
      Security)}{Fall $2023$}{Professor: Limin Jia}{}
  \item Topics: Cross-Site Scripting, Request Forgery, browser policy.
  \end{rSubsection}

  \begin{rSubsection}{Teaching Assistant for 18-732 (Software
      Security)}{Spring $2021$}{Professor: Bryan Parno}{}
  \item Topics: Control-Flow Integrity, Software Fault Isolation,
    Fuzzing, Model-Checking, Formal Verification.
  \end{rSubsection}

  \begin{rSubsection}{Artifact Evaluation for VMCAI 2021}{October $2020$}
    {}{}
  \item Evaluated 4 artifacts for the VMCAI (Verification, Model
    Checking, and Abstract Interpretation) 2021.
  \end{rSubsection}

  \begin{rSubsection}{Student Volunteer for CSF 2020}{June $2020$}
    {}{}
  \item Helped manage virtual conferencing for CSF
    (Computer Security Foundations Symposium) 2020.
  \end{rSubsection}
\end{rSection}

\end{document}
