\documentclass{resume} % Use the custom resume.cls style
% Document margins
\usepackage[utf8]{inputenc}
\usepackage[left=0.2in,top=0.6in,right=0.2in,bottom=0.6in]{geometry}
\usepackage[T1]{fontenc}
\usepackage{titling}
\usepackage{atbegshi}
\usepackage{hyperref}
\usepackage{nopageno}
\thispagestyle{empty}
% http://ctan.org/pkg/atbegshi
\AtBeginDocument{\AtBeginShipoutNext{\AtBeginShipoutDiscard}}

\setlength{\droptitle}{-0.6in}   % reduce space before title

\title{\bf Yoshiki Takashima}
\author{\href{ytakashi@andrew.cmu.edu}{ytakashi@andrew.cmu.edu}}
\date{4720 Forbes Ave, RMCIC-2119C. Pittsburgh PA 15213}

\begin{document}
\maketitle

\begin{rSection}{Research Interest}
  I want to helping developers build more robust software by
  leveraging automated, language-aware solutions for software testing and
  verification. My recent works focus on testing and verifying Rust
  programs by leveraging Large-Language Models and Program Synthesis.
\end{rSection}

\begin{rSection}{Education}
  \begin{rSubsection}{Carnegie Mellon University}{$2019 -$
      Present}{PhD Student in Electrical and Computer
      Engineering}{Pittsburgh, PA}
  \item Co-Advised by: Prof. Limin Jia and Prof. Corina P\u{a}s\u{a}reanu.
  \item Proposed Thesis: \textit{Testing and Verifying Rust's Next Mile}. Expected Apr. 2024.
  \end{rSubsection}

  \begin{rSubsection}{UC San Diego}{$2017 -
      2019$}{
      \href{https://catalog.ucsd.edu/curric/MATH-ug.html}
      {BS Mathematics - Comp. Sci. (MA30)}}{La Jolla, CA}
  \item GPA $3.95/4.00$. Credits transferred from Santa Monica College
    and West Los Angeles College.
  \end{rSubsection}

  % \begin{rSubsection}{Santa Monica College}{$2015 -
  %     2017$}{IGETC Transfer Certificate}{Santa Monica, CA}
  % \item GPA: $3.86/4.00$
  % \item High school concurrent enrollment.
  % \end{rSubsection}

  % \begin{rSubsection}{West Los Angeles College}{$2014$}{Credits
  %     Transferred}{Los Angeles, CA}
  % \item GPA: $4.00/4.00$
  % \item Concurrent enrollment.
  % \end{rSubsection}
\end{rSection}

\begin{rSection}{Employment}

  \begin{rSubsection}{Applied Scientist Intern (Automated Reasoning),
      Amazon Web Services}{Summer $2023$} {}{}
  \item Developed an LLM-based universal transpiler into Rust that
    guarantees correct translation by equivalence-checking the
    candidate translation against a Web Assembly-based trusted oracle.
  \end{rSubsection}

  \begin{rSubsection}{Applied Scientist Intern (Automated Reasoning),
      Amazon Web Services}{Summer $2022$} {}{}
  \item Developed PropProof, a library that automatically converts
    Property-Based Tests into model-checking by replacing
    random values with optimized symbolic models. Integrated it into
    GitHub CI of AWS Open Source Project \texttt{PROST}.
  \end{rSubsection}
\end{rSection}

\begin{rSection}{Publications}
  \begin{rPubsection}{Crabtree: Rust API Test Synthesis Guided by
      Coverage and Type}{Under Submission: PLDI'24}{\textbf{Yoshiki
        Takashima}, Chanhee Cho, Ruben Martins, Limin Jia, Corina
      S. P\u{a}s\u{a}reanu.}{}
  \end{rPubsection}

  \begin{rPubsection}{VERT: Verified Equivalent Rust Transpilation
      with Large Language Models}{Under Submission: FSE'24}{Aidan Yang*,
      \textbf{Yoshiki Takashima*}, Brandon Paulsen, Joey Dodds, Daniel
      Kroening}{}

    {\footnotesize *Equal Contribution}
  \end{rPubsection}

  \begin{rPubsection}{Automatically Enforcing Rust Trait
      Properties}{VMCAI'24}{Twain Byrnes, \textbf{Yoshiki Takashima}, Limin
      Jia.}{}
  \end{rPubsection}

  \begin{rPubsection}{PropProof: Free Model-Checking Harnesses from
      PBT}{ESEC/FSE’23 Industry Track}{\textbf{Yoshiki Takashima}.}{}

    {\footnotesize
      \href{https://github.com/YoshikiTakashima/propproof}
      {https://github.com/YoshikiTakashima/propproof}}
  \end{rPubsection}

  \begin{rPubsection}{Mariposa: Measuring SMT Instability in Automated
      Program Verification}{FMCAD'23}{Yi Zhou, Jay Bosamiya, \textbf{Yoshiki
      Takashima}, Jessica Li, Marijn Heule, Bryan Parno.}{}

    % {\footnotesize
    %   \href{https://github.com/secure-foundations/mariposa}
    %   {https://github.com/secure-foundations/mariposa}}
  \end{rPubsection}

  \begin{rPubsection}{SyRust: Automatic Testing of Rust Libraries with
      Semantic-Aware Program Synthesis}{PLDI'21}{\textbf{Yoshiki Takashima},
      Ruben Martins, Limin Jia, and Corina S. P\u{a}s\u{a}reanu.}{}

    {\footnotesize Paper and video available here:
      \href{https://doi.org/10.1145/3453483.3454084}
      {https://doi.org/10.1145/3453483.3454084}. Discovered bugs lead
      to
      \href{https://nvd.nist.gov/vuln/detail/CVE-2020-15254}{CVE-2020-15254}}
  \end{rPubsection}

  \begin{rPubsection}{VeriSketch: Synthesizing Secure Hardware Designs
      with Timing-Sensitive Information Flow
      Properties}{CCS'19}{Armaiti Ardeshiricham, \textbf{Yoshiki
        Takashima~(presenter)}, Sicun Gao, Ryan Kastner.}{}

    {\footnotesize Paper and artifact:
      \href{https://dl.acm.org/doi/abs/10.1145/3319535.3354246}
      {https://doi.org/10.1145/3319535.3354246}}
  \end{rPubsection}
\end{rSection}

\begin{rSection}{Awards and Fellowships}

  \begin{rSubsection}{Amazon Research Award 2022: Enabling One-Line Rust
      Verification with Program Synthesis}{}{}{}
  \item Written with advisors Limin Jia and Corina P\u{a}s\u{a}reanu.
  \end{rSubsection}

  \begin{rSubsection}{CMU ECE Prabhu and Poonam Goel Graduate
      Fellowship}{$2021 - 2022$}{}{}
  \item Internal fellowship awarded by the CMU Electrical and Computer
    Engineering Department.
  \end{rSubsection}
\end{rSection}

\begin{rSection}{Teaching and Service}
  \begin{rSubsection}{Teaching Assistant for CMU 18-636 (Web
      Security)}{Fall $2023$}{Professor: Limin Jia}{}
  \item Topics: Cross-Site Scripting, Request Forgery, browser policy.
  \end{rSubsection}

  \begin{rSubsection}{Teaching Assistant for CMU 18-732 (Software
      Security)}{Spring $2021$}{Professor: Bryan Parno}{}
  \item Topics: Control-Flow Integrity, Software Fault Isolation,
    Fuzzing, Model-Checking, Formal Verification.
  \end{rSubsection}

  \begin{rSubsection}{Artifact Evaluation for VMCAI 2021}{October $2020$}
    {}{}
  \item Evaluated 4 artifacts for the VMCAI (Verification, Model
    Checking, and Abstract Interpretation) 2021.
  \end{rSubsection}

  \begin{rSubsection}{Student Volunteer for CSF 2020}{June $2020$}
    {}{}
  \item Helped manage virtual conferencing for CSF
    (Computer Security Foundations Symposium) 2020.
  \end{rSubsection}
\end{rSection}

\begin{rSection}{Skills}
  \begin{rSubsection}{}{}{}{}
  \item Programming Language: Rust, SQL, Python, R, Java, Dafny, C, C++, Scala, OCaml.
  \item Operating System: Linux, Mac.
  \item Document Writing: LaTeX, Microsoft Word, HTML.
  \item Human Language: Fluent in English and Japanese.
  \end{rSubsection}
\end{rSection}
\end{document}

%%% Local Variables:
%%% mode: latex
%%% TeX-master: t
%%% End:
