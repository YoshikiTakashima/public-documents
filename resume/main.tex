\documentclass{resume} % Use the custom resume.cls style
% Document margins
\usepackage[utf8]{inputenc}
\usepackage[left=0.75in,top=0.6in,right=0.75in,bottom=0.6in]{geometry}
\usepackage[T1]{fontenc}
\usepackage{titling}
\usepackage{atbegshi}
\usepackage{hyperref}
\usepackage{nopageno}
\thispagestyle{empty}
% http://ctan.org/pkg/atbegshi
\AtBeginDocument{\AtBeginShipoutNext{\AtBeginShipoutDiscard}}

% \sloppy

\title{\bf Yoshiki Takashima}
\author{\href{ytakashi@andrew.cmu.edu}{ytakashi@andrew.cmu.edu}}
\date{4720 Forbes Ave, RMCIC-2119C. Pittsburgh PA 15213}

\begin{document}
\maketitle

\begin{rSection}{Research Interest}
  I am interested in helping developers build more robust software by
  leveraging automated, language-aware solutions for testing and
  verification. Recent work includes automatic testing of Rust libraries.
\end{rSection}

\begin{rSection}{Education}
  \begin{rSubsection}{Carnegie Mellon University}{$2019 -$
      Present}{PhD Student in Electrical and Computer
      Engineering}{Pittsburgh, PA}
  \item Co-Advised by: Prof. Limin Jia and Prof. Corina Pasareanu.
  % \item QPA 3.62/4.00
  \end{rSubsection}

  \begin{rSubsection}{UC San Diego}{$2017 -
      2019$}{
      \href{https://www.math.ucsd.edu/~handbook/undergraduate/ma30-math-computer-science-b-s/}
      {BS Mathematics - Comp. Sci.}}{La Jolla, CA}
  \item GPA: $3.95/4.00$
  \end{rSubsection}

  \begin{rSubsection}{Santa Monica College}{$2015 -
      2017$}{IGETC Transfer Certificate}{Santa Monica, CA}
  \item GPA: $3.86/4.00$
  \item High school concurrent enrollment.
  \end{rSubsection}

  \begin{rSubsection}{West Los Angeles College}{$2014$}{Credits
      Transferred}{Los Angeles, CA}
  \item GPA: $4.00/4.00$
  \item Concurrent enrollment.
  \end{rSubsection}
\end{rSection}

\begin{rSection}{Publications}

  \begin{rSubsection}{}{$2021$}{}{}
  \item \textbf{SyRust: Automatic Testing of Rust Libraries
    with Semantic-Aware Program Synthesis}.
    Yoshiki Takashima, Ruben Martins, Limin Jia, and Corina
    S. P\u{a}s\u{a}reanu.  In Proceedings of the 42nd
    ACM SIGPLAN International Conference on Programming Language
    Design and Implementation \textit{(PLDI’21)}.

    Paper, video, and artifact available here:
    \href{https://doi.org/10.1145/3453483.3454084}
    {https://doi.org/10.1145/3453483.3454084}
  \end{rSubsection}

  \begin{rSubsection}{}{$2019$}{}{}
  \item \textbf{VeriSketch: Synthesizing Secure Hardware Designs with
      Timing-Sensitive Information Flow Properties}.  Armaiti
    Ardeshiricham, Yoshiki Takashima (\textbf{presenter}), Sicun Gao,
    Ryan Kastner. In Proceedings of the 2019 ACM SIGSAC Conference on
    Computer and Communications Security~\textit{(CCS'19)}.

    Paper and video available here:
    \href{https://dl.acm.org/doi/abs/10.1145/3319535.3354246}
    {https://doi.org/10.1145/3319535.3354246}
  \end{rSubsection}
\end{rSection}

\begin{rSection}{Teaching and Service}

  \begin{rSubsection}{Teaching Assistant for 18-732 (Software
      Security)}{Spring $2021$}{Professor: Bryan Parno}{}
  \item This ECE course teaches students methodologies for developing secure
    software, ranging from static analysis to formal verification using Dafny.
  \item Maintained infrastructure, conducted office hours and recitations.
  \end{rSubsection}

  \begin{rSubsection}{Artifact Evaluation for VMCAI 2021}{October $2020$}
    {}{}
  \item Evaluated 4 artifacts for the VMCAI (Verification, Model
    Checking, and Abstract Interpretation) 2021.
  \end{rSubsection}

  \begin{rSubsection}{Student Volunteer for CSF 2020}{June $2020$}
    {}{}
  \item Helped manage virtual conferencing for CSF
    (Computer Security Foundations Symposium) 2020.
  \end{rSubsection}
\end{rSection}

%% Ongoing Research
\begin{rSection}{Ongoing Research}
  \begin{rSubsection}{Improving Test Case Quality for SyRust}{}{}{}
  \item Extend SyRust with heuristics to produce higher-quality test
    cases that efficiently achieve higher code coverage with less
    computational resources.
  \end{rSubsection}
  \begin{rSubsection}{Robust Detection of Reflected Cross-Site
      Scripting (XSS)}{}{}{}
  \item Built a suite of tools for Reflected XSS detection,
    leveraging a modified browser, user interface fuzzing, and search
    engine mining for better coverage and improved vulnerability
    detection.
  \end{rSubsection}
\end{rSection}

%% Personal Projects

\begin{rSection}{Personal Projects}
  \begin{rSubsection}{Co-Opting Optimization Techniques for Finding
      Provable Global Minimum}{$2019$}{}{}
  \item GitHub Link:
    \href{https://github.com/YoshikiTakashima/dreal-cmake-example-project}
    {https://github.com/YoshikiTakashima/dreal-cmake-example-project}
  \end{rSubsection}
  \begin{rSubsection}{Recognition of Hand-Written Automata Diagrams with
      Machine Learning}{$2018$}{}{}
  \item GitHub Link:
    \href{https://github.com/YoshikiTakashima/GradeRegular}
    {https://github.com/YoshikiTakashima/GradeRegular}
  \end{rSubsection}
\end{rSection}
\end{document}
